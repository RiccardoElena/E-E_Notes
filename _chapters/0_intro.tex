\begin{abstract}
  Questo file contiene appunti e materiali di studio relativi al corso di Encoding \& Encryption tenuto dal Prof.~A.~De Luca alla Università degli Studi Federico II di Napoli nell'A.A.~25/26.
  
  Questo corso tratterà i fondamenti della codifica di sorgente e canale nell'ottica della teoria dell'informazione di Shannon, nonché i principi e le tecniche di crittografia moderna.

  \textbf{Questi appunti sono destinati a supportare gli studenti nel loro percorso di apprendimento e a fornire una risorsa di riferimento per gli argomenti discussi durante il corso, ma non sostituiscono in alcun modo il materiale ufficiale e le lezioni del docente.}
  Questo documento è prodotto da studenti, per studenti, e potrebbe dunque contenere errori o imprecisioni. Feedback e correzioni sono ben accetti e incentivati, e possono essere inviati tramite pull request o issue nella \href{https://github.com/RiccardoElena/E-E_Notes}{repository GitHub del documento}.

  A tal proposito, nel documento saranno presenti alcuni contenuti non provenienti direttamente dalle lezioni, ma aggiunti o modificati dagli autori per migliorare la comprensibilità o l'approfondimento degli argomenti trattati.
  Ci impegniamo a verificare col docente ogni modifica sostanziale al materiale originale, ma in caso tale verifica non fosse ancora avvenuta, verranno inseriti degli indicatori per segnalare le eventuali divergenze dal materiale ufficiale.
  In particolare:
  \begin{itemize}
    \item I contenuti aggiunti o modificati in maniera sostanziale dagli autori saranno segnalati con il simbolo \warningsymbol{}
    \item I contenuti che invece hanno subito esclusivamente riorganizzazione strutturale o di formattazione verranno segnalati col simbolo \infosymbol{}
  \end{itemize}
  
  % Firma autori / link repo (posizionata come firma in basso a destra)
  \begin{flushright}
    \vspace{2cm}
    \textbf{Gli Autori:}\\
    \href{https://github.com/RiccardoElena}{Riccardo Elena}, \href{https://github.com/TheFabbest}{Fabrizio Apuzzo} \\
  \end{flushright}
\end{abstract}