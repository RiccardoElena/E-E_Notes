\chapter{Lezione 7 --- Draft}


\begin{theorem}[Schützenberger]
  Sia \(X\) codice su \(A\). Allora:
  \begin{enumerate}
    \item \(X\) massimale \(\implies\) \(X\) completo\label{item:schutz1}
    \item \(X\) completo e non denso \(\implies\) \(X\) massimale\label{item:schutz2}
  \end{enumerate}
\end{theorem}
\begin{proof}
  \paragraph{\ref{item:schutz1}.}
  Mostriamo che se \(X\) non è completo allora non è massimale.
  Il caso \(\# A = 1\) è banale, quindi supponiamo \(\# A \geq 2\).
  Sia \(y \in A^{*} \) che non si completa in \(X^{*}\), senza bordi\footnote{Se \(y\) ha bordi, possiamo costruire facilmente un'altra parola \(y'\) senza bordi partendo da \(y\) che non si completa in \(X^{*}\)}.


  Mostriamo che \(Y = X \cup \{y\}\) è un codice.
  Sia \(y_1 y_2 \cdots y_n = y_1' y_2' \cdots y_m', y_1\neq y_1'\) una doppia fattorizzazione di una parola di \(Y^*\).
  Essendo \(X\) codice \(y\) deve occorrere sia a sinistra che a destra.\footnote{Non solo in un lato poiché altrimenti l'altro sarebbe un modo di completare \(y\) in \(X^*\).}
  Poiché \(y\) non si completa in \(X^*\) esistono \(i\) e \(j\) tali che \(y_i = y_j' = y\), minimali.

  \TODO{Inserire figura 1}

  Analizziamo i possibili casi:
  \begin{enumerate}
    \item \(y_i\) e \(y_j'\) non si sovrappongono: allora \(y_i\) occorre a sinistra di \(y_j'\) o viceversa. Allora \(y\) si completa in \(X^*\), assurdo.
    \item \(y_i\) e \(y_j'\) si sovrappongono parzialmente.
      \TODO{Inserire figura 2}
      Allora \(y\) avrebbe bordo, assurdo.
      \item \(y_i\) e \( y_j'\) si sovrappongono completamente: allora \(y_1y_2 \cdots y_{i-1} = y_1'y_2' \cdots y_{j-1}'\), che però sono tutte parole in \(X^*\), ottenendo una doppia fattorizzazione di una parola in \(X^*\), che però è un codice, assurdo. 
  \end{enumerate}
\end{proof}

\begin{note}
  L'ipotesi \emph{non denso} in~\ref{item:schutz2} è necessaria. Infatti il codice \(Z = \{a^{|u|+1}bu | u \in {\{a,b\}}^*\}\) è un codice denso (e quindi completo), ma non massimale.
  Infatti \(Z\cup\{b\}\) è ancora un codice (prefisso).
\end{note}

\begin{corollary}
  Sia \(X \subseteq A^+\) codice non denso. Allora le seguenti affermazioni sono equivalenti:
  \begin{enumerate}
    \item \(X\) è massimale
    \item \(X\) è completo
    \item \(\forall \mu\) distribuzione positiva su \(A\): \(\mu(X) = 1\)
    \item \(\exists \mu\) distribuzione positiva su \(A\): \(\mu(X) = 1\)
  \end{enumerate}
\end{corollary}
\begin{proof}
  Si ha che \(1 \iff 2\) per il teorema di Schützenberger, \(2 \implies 3\) viene dalla dimostrazione del punto~\ref{item:schutz2} del teorema di Schützenberger, \(3 \implies 4\) è banale e \(4 \implies 1\) lo abbiamo visto più volte.
\end{proof}

\begin{proposition}
  Sia \(X\subseteq A^*\) finito e completo, e \(\mu\) distribuzione positiva su \(A\) tale che \(\mu(X) = 1\). Allora \(X\) è un codice (massimale).
\end{proposition}

\begin{proof}
  Dato \(n \geq 1\), \(X^n\) è completo (se \(w\) si completa in \(X^*\) allora si completa in \({(X^n)}^*\))
  Per il lemma di Marcus-Schützenberger, \TODO{aggiungi ref al teorema una volta inserito}
  \(1\leq \mu(X^n) = {\mu(X)}^n = 1^n = 1\)
  Quindi \(\forall n \geq 1: \mu(X^n) = 1 = {\mu(X)}^n\). Dunque \(X\) è un codice. RISULTATO PRECEDENTE
\end{proof}

\begin{proposition}
  Sia \(X \subseteq A^*\) finito e completo. Allora \(\forall a \in A \exists n \geq 1 : a^n \in X\). Tale \(n\) è unico se \(X\) è un codice.
\end{proposition}
\begin{proof}
  Sia \(L = \max_{x \in X}\abs{x}\) e \(a \in A\). Essendo \(X\) completo, \(a^{2L}\) si completa in \(X^*\), quindi esistono \(\lambda, \rho \in A^*: \lambda a^{2L} \rho \in X^*\).
  \TODO{Inserire figura 3}
  Essendo che ogni parola in \(X\) ha lunghezza al più \(L\), esiste un \(x_i\) della fattorizzazione di \(\lambda a^{2L} \rho\) fattore della parola \(a^{2L}\).
  \TODO{Inserire figura 4}
  Una tale fattorizzazione infatti porterebbe ad avere \(x_i,x_{i+1} \in X\) tali che \(\abs{x_i}+\abs{x_{i+1}} > 2L\), assurdo.
  Da ciò segue che tale \(x_i\) è della forma \(a^n\). Questo è unico poiché se esistesse \(m\neq n\) tale che \(a^m \in X\), allora la parola \(a^{n+m}\) avrebbe due fattorizzazioni distinte in \(X^*\), assurdo se \(X\) è un codice.
\end{proof}

\begin{example}
  Il codice prefisso \(X = \set{aa,aba,cbb,ba,bbc}\) è facilmente verificabile non essere massimale. Infatti, questo codice non contiene potenze di \(b\), e dunque dalla proposizione precedente non è completo, e quindi non è massimale.
\end{example}

\begin{definition}
  Dato \(X \subseteq A^*\) codice, chiamiamo \keyword{completamento} di \(X\) un codice \(Y \supseteq X\) nello stesso alfabeto, massimale.
\end{definition}

Da questa definizione sorgono due domande:
\begin{enumerate}
  \item Dato un codice \(X\) qualsiasi, esiste sempre un completamento? (Spoiler, sì)
  \item Dato un codice \(X\) finito, esiste sempre un completamento finito? (Spoiler, no)
    Esempio noto fornito da Markov ha mostrato che dato \(A = \set{a,b}\), il codice \(X = \set{a^5,ab,ba^2,b}\) non ha completamenti finiti.
  \item A questo punto, \emph{quali} codici finiti ammettono completamento finito?
    In generale, la caratterizzazione completa è un problema aperto, ma:
    \begin{itemize}
      \item I codici finiti prefissi ammettono sempre completamenti finiti. (Tra essi in particolare i codici su \(A\) con \(\abs{A} = 1\)).
      \item Se \(\#X = 2\), allora \(X\) ammette un completamento finito (Restivo).
      \item Se \(\#X \geq 4\), in generale non ammette un completamento finito.
      \item Se \(\#X = 3\), il problema è aperto.
    \end{itemize}
\end{enumerate}

Vediamo dunque la dimostrazione dell'esistenza di un completamento per ogni codice citato in precedenza.
\begin{proposition}
  Sia \(X \subseteq A^*\) codice. Allora esiste un completamento di \(X\).
\end{proposition}
\begin{proof}
  Se \(X\) è massimale, allora \(X = X_0\) è un completamento di sé stesso.
  Altrimenti, esiste \(w_1 \in A^*\setminus X\) di lunghezza minima tale che \(X_1 = X \cup \{w_1\}\) è ancora un codice.
  Se \(X_1\) è massimale, abbiamo finito.
  Se iterando questo procedimento otteniamo \(X_k = X_{k-1} \cup \set{w_k}\) massimale, abbiamo finito.
  Altrimenti, otteniamo una successione infinita di parole \(s = \set{w_n}\). Essendo però l'alfabeto finito, esistono parole in \(s\) arbitrariamente lunghe.
  Mostriamo dunque che \(Y = \bigcup_{k = 0}^{\infty} X_k\) è un codice massimale.
  È codice poiché se avessimo una doppia fattorizzazione \(y_1 y_2 \cdots y_n = y_1' y_2' \cdots y_m'\) avremmo che ogni \(y_i\) e \(y_j'\) appartengono a qualche \(X_k\). Essendo che \(X_0 \subseteq X_1 \subseteq X_2 \subseteq \cdots\), esiste \(n\) tale che tutti i \(y_i\) e \(y_j'\) appartengono a \(X_n\). Ma allora avremmo una doppia fattorizzazione in \(X_n\), assurdo.
  Per quanto riguarda la massimalità, se avessimo \(w \in A^* \setminus Y\) tale che \(Y \cup \{w\}\) è ancora un codice, allora esisterebbe \(m\) tale che \(\abs{w_m} > \abs{w}\).
  Ma allora \(X_m \cup \set{w}\subseteq Y \cup \set{w}\) sarebbe ancora codice, contraddicendo la minimalità di \(w_m\).
\end{proof}

\begin{definition}[Sorgente (discreta e a memoria zero)]
  Chiamiamo \keyword{Sorgente discreta e a memoria zero} variabile aleatoria discreta, identificabile come una coppia \(S = (\mathcal{S},p)\), con \(\mathcal{S}\) alfabeto sorgente e \(p\) distribuzione su \(\mathcal{S}\).
\end{definition}
\begin{definition}[Codifica]
  Definiamo \keyword{codifica} un morfismo iniettivo \(\phi: \mathcal{S}^* \to A^*\) con \(A\) alfabeto (di codice).
  Il \keyword{codice} relativo a questa codifica è \(X = \phi(\mathcal{S})\), ovvero l'immagine di \(\mathcal{S}\) sotto \(\phi\).
\end{definition}
\begin{definition}[Costo di codifica]
  Chiamiamo \keyword{costo} di \(phi\) la quantità
  \[C(X,\phi) = \sum_{s \in \mathcal{S}} p(s) \abs{\phi(s)}\]
  ovvero la media pesata sulla distribuzione \(p\) delle lunghezze delle parole codificate.
  Il \keyword{costo assoluto} di un codice \(X\) sarà
  \[C(X) = \min_{\phi: \phi(\mathcal{S}) \leftrightarrow  X} C(X,\phi)\]
\end{definition}

\begin{example}
  Sia \(\mathcal{S} = \set{s_1,s_2,s_3}\), \(A = \set{a,b}\), \(X = \set{a,ba,bb}\).
  Se \(p(s_1) = 1/2, p(s_2) = p(s_3) = 1/4\), e inoltre \(\phi(s_1) = ba, \phi(s_2) = a, \phi(s_3) = bb\), si ha che
    \[C(X,\phi) = \frac{7}{4} > C(X) = \frac{3}{2}\]
\end{example}